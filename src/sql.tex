\part{SQL}
\textbf{SQL} (Structured Query Language) is case-insensitive, but keywords are uppercase by convention.

The most common data types are \code{INTEGER}, \code{VARCHAR(n)}, \code{BOOLEAN}, and \code{DATE}.

\section{Operations and Syntax}

\subsection{Table Operations}
\begin{lstlisting}
CREATE TABLE table_name (
    -- define attributes (columns)
    <attribute> <type> [<column_constraint>],
    <attribute> <type> [CONSTRAINT <name> <cstr.>],
    ...
    -- define optional table constraints
    [<table_constraint>],
    -- constraints can be named
    [CONSTRAINT <name> <table_constraint>],
    ... /* alternative comment syntax */
    [<table_constraint>] -- no trailing comma
);

ALTER TABLE <table>
    [ALTER / ADD / DROP] [COLUMN / CONSTRAINT]
    <attribute / name>
    <changes>;

DROP TABLE
    [IF EXISTS] -- no error if table doesn't exist
    <table>, ... -- drop multiple tables at once
    [CASCADE]; -- also drop referencing tables
\end{lstlisting}

\subsection{Integrity Constraints}
Constraints are specified in the \code{CREATE TABLE} statement,
and reject insertions if the condition evaluates to \code{false} (\textbf{principle of rejection}).

\textbf{Column constraints} are defined on a per-column basis, while \textbf{table constraints} are defined on the table as a whole.

Constraints can be \textit{named}, or \textit{unnamed}, in which case the DBMS will generate a name.

There are 5 types of constraints that can be placed on a table $R$ and/or any attribute $a$:
\begin{enumerate*}
    \keyitem*{\code{NOT NULL}}{$\forall r \in R: r_a \not\equiv \code{NULL}$}
    \keyitem*{\code{UNIQUE}}{$\forall r_1, r_2 \in R: (r_1 \equiv r_2) \lor (\exists r_{1_a} <> r_{2_a})$}
    \keyitem*{\code{PRIMARY KEY}}{equivalent to \code{UNIQUE} and \code{NOT NULL}}
    \keyitem{\code{FOREIGN KEY ($a, ...$) REFERENCES $R'$($a', ...$)}}{$\forall r \in R: (\forall a: r_a \in R_{a'}') \lor (\code{NULL} \in r)$}
    \keyitem*{\code{CHECK($c$)}}{$c$ does not evaluate to \code{false}}
\end{enumerate*}

For \code{UNIQUE}, note that \code{NULL} $<>$ \code{NULL} evaluates to \code{NULL},
such that duplicate insertions of \code{NULL} are not rejected.

For \code{FOREIGN KEY}, $R'$ must be a valid table name. \\
\code{SET NULL}, \code{SET DEFAULT}, or \code{CASCADE} can be specified as the action to take when a referenced row is deleted.

\code{CASCADE} will delete all referencing rows, propagating the deletion, which may significantly affect performance.

For \code{CHECK}, $c$ must be a boolean expression scoped to the table on individual rows.

\subsection{Row Operations}
\begin{lstlisting}
INSERT INTO <table> [(attribute, ...)]
    VALUES -- whitespace and newlines optional
        (value, ...),
        (value, ...),
        ... -- alternatively, all on one line
        (value, ...);
\end{lstlisting}

All values to insert must have the same shape as either the table, or, if specified, the attribute list.

Attributes in the attribute list can be in any order.

Attributes missing from the attribute list will have their values set to \code{NULL}, or, if specified in the schema, to their default value.

\begin{lstlisting}
DELETE FROM <table> [WHERE <condition>];
\end{lstlisting}

The condition only deletes rows which evaluate to \code{TRUE} (\textbf{principle of acceptance}).

If the condition is omitted, all rows will be deleted.

The condition must be a boolean expression scoped to to the table on individual rows.

\subsection{Deferrable Constraints}
\begin{lstlisting}
BEGIN; -- start a transaction
-- perform operations
COMMIT; -- commit (end) the transaction
\end{lstlisting}

Constraints are checked immediately at the end of every SQL statement (these end with a semicolon) and transaction --- violation performs a rollback.

When defining a constraint, three deferments can be specified:
\begin{enumerate*}
    \keyitem{\code{NOT DEFERRABLE}}{if unspecified, this is the default}
    \keyitem{\code{DEFFERABLE INITIALLY IMMEDIATE}}{constraint is checked immediately, but can be deferred later}
    \keyitem{\code{DEFFERABLE INITIALLY DEFERRED}}{constraint is deferred by default}
\end{enumerate*}

\code{DEFFERABLE INITIALLY IMMEDIATE} gives the \textit{option} to defer the constraint checking by adding the following line in a transaction, i.e.:

\begin{lstlisting}
BEGIN;
SET CONSTRAINTS <name> DEFERRED; -- add this line
-- perform operations
COMMIT;
\end{lstlisting}

Deferring a constraint is necessary when the constraint depends on a row that is inserted later in the transaction, e.g. circular foreign key constraints.