\part{SQL}
\textbf{SQL} (Structured Query Language) is case-insensitive, but keywords are uppercase by convention.

The most common data types are \code{INTEGER}, \code{VARCHAR(n)}, \code{BOOLEAN}, and \code{DATE}.

\section{Operations and Syntax}

\subsection{Table Operations}
\begin{lstlisting}
CREATE TABLE table_name (
    -- define attributes (columns)
    <attribute> <type> [<column_constraint>],
    <attribute> <type> [CONSTRAINT <name> <cstr.>],
    ...
    -- define optional table constraints
    [<table_constraint>],
    -- constraints can be named
    [CONSTRAINT <name> <table_constraint>],
    ... /* alternative comment syntax */
    [<table_constraint>] -- no trailing comma
);

ALTER TABLE <table>
    [ALTER / ADD / DROP] [COLUMN / CONSTRAINT]
    <attribute / name>
    <changes>;

DROP TABLE
    [IF EXISTS] -- no error if table doesn't exist
    <table>, ... -- drop multiple tables at once
    [CASCADE]; -- also drop referencing tables
\end{lstlisting}

\subsection{Integrity Constraints}
Constraints are specified in the \code{CREATE TABLE} statement,
and reject insertions if the condition evaluates to \code{false} (\textbf{principle of rejection}).

\textbf{Column constraints} are defined on a per-column basis, while \textbf{table constraints} are defined on the table as a whole.

Constraints can be \textit{named}, or \textit{unnamed}, in which case the DBMS will generate a name.

There are 5 types of constraints that can be placed on a table $R$ and/or any attribute $a$:
\begin{enumerate*}
    \keyitem*{\code{NOT NULL}}{$\forall r \in R: r_a \not\equiv \code{NULL}$}
    \keyitem*{\code{UNIQUE}}{$\forall r_1, r_2 \in R: (r_1 \equiv r_2) \lor (\exists r_{1_a} <> r_{2_a})$}
    \keyitem*{\code{PRIMARY KEY}}{equivalent to \code{UNIQUE} and \code{NOT NULL}}
    \keyitem{\code{FOREIGN KEY ($a, ...$) REFERENCES $R'$($a', ...$)}}{$\forall r \in R: (\forall a: r_a \in R_{a'}') \lor (\code{NULL} \in r)$}
    \keyitem*{\code{CHECK($c$)}}{$c$ does not evaluate to \code{false}}
\end{enumerate*}

For \code{UNIQUE}, note that \code{NULL} $<>$ \code{NULL} evaluates to \code{NULL},
such that duplicate insertions of \code{NULL} are not rejected.

For \code{FOREIGN KEY}, $R'$ must be a valid table name. \\
\code{SET NULL}, \code{SET DEFAULT}, or \code{CASCADE} can be specified as the action to take when a referenced row is deleted.

\code{CASCADE} will delete all referencing rows, propagating the deletion, which may significantly affect performance.

For \code{CHECK}, $c$ must be a boolean expression scoped to the table on individual rows.

\subsection{Row Operations}
\begin{lstlisting}
INSERT INTO <table> [(attribute, ...)]
    VALUES -- whitespace and newlines optional
        (value, ...),
        (value, ...),
        ... -- alternatively, all on one line
        (value, ...);
\end{lstlisting}

All values to insert must have the same shape as either the table, or, if specified, the attribute list.

Attributes in the attribute list can be in any order.

Attributes missing from the attribute list will have their values set to \code{NULL}, or, if specified in the schema, to their default value.

\begin{lstlisting}
DELETE FROM <table> [WHERE <condition>];
\end{lstlisting}

The condition only deletes rows which evaluate to \code{TRUE} (\textbf{principle of acceptance}).

If the condition is omitted, all rows will be deleted.

The condition must be a boolean expression scoped to to the table on individual rows.

\subsection{Deferrable Constraints}
\begin{lstlisting}
BEGIN; -- start a transaction
-- perform operations
COMMIT; -- commit (end) the transaction
\end{lstlisting}

Constraints are checked immediately at the end of every SQL statement (these end with a semicolon) and transaction --- violation performs a rollback.

When defining a constraint, three deferments can be specified:
\begin{enumerate*}
    \keyitem{\code{NOT DEFERRABLE}}{if unspecified, this is the default}
    \keyitem{\code{DEFFERABLE INITIALLY IMMEDIATE}}{constraint is checked immediately, but can be deferred later}
    \keyitem{\code{DEFFERABLE INITIALLY DEFERRED}}{constraint is deferred by default}
\end{enumerate*}

\code{DEFFERABLE INITIALLY IMMEDIATE} gives the \textit{option} to defer the constraint checking by adding the following line in a transaction, i.e.:

\begin{lstlisting}
BEGIN;
SET CONSTRAINTS <name> DEFERRED; -- add this line
-- perform operations
COMMIT;
\end{lstlisting}

Deferring a constraint is necessary when the constraint depends on a row that is inserted later in the transaction, e.g. circular foreign key constraints.


\section{Queries}
SQL by default does not eliminate duplicate rows (without \code{DISTINCT}) nor does it have a fixed row ordering (\textbf{order-independent}) (without \code{ORDER BY}).

The basic query syntax is as follows:
\begin{lstlisting}
SELECT [DISTINCT] <attributes> -- * or a1, a2, ...
    FROM <tables> -- r1, r2, ...
    WHERE <conditions> -- c
    ORDER BY <attributes> <ASC / DESC>
    OFFSET <n>
    LIMIT <n>; -- ; only if nothing else follows
[[UNION / INTERSECT / EXCEPT] [ALL]]
    ... -- more queries
\end{lstlisting}

\code{DISTINCT} uses \code{IS DISTINCT FROM} to compare rows, which is different from \code{=}.
The latter returns \code{NULL} if either argument is \code{NULL}, while the former treats \code{NULL} like any other value.

Condition evaluation is based on the \textbf{principle of acceptance} --- rows are included if and only if the condition evaluates to \code{TRUE}.

The above query up to line 3 is equivalent to the following relational algebra expression:

$ \pi_{[a_1, a_2, \dots]} \left(\sigma_{[c]} (r_1 \times r_2 \times ...) \right) $

\begin{defn}{\code{SELECT} and \code{FROM} clauses --- \code{AS} expression}
    Rename columns or tables (within the query scope) with \code{AS}, and even operate on entire columns with several functions:
    \begin{itemize}
        \keyitem*{mathematical}{\code{+ - * / \% |/ \^}}
        \keyitem*{logical}{\code{AND, OR, NOT}}
        \keyitem*{string}{\code{||, LOWER(s), UPPER(s)}}
        \keyitem*{date}{\code{+, NOW()}}
    \end{itemize}

    \begin{lstlisting}
SELECT <regular attributes> -- a1, a2, ...
    [<expression> AS <new attribute>], -- ...
FROM <regular tables>,
    [<old table name> [AS] <new table name>];
    \end{lstlisting}

    Renaming tables is necessary if the same table appears more than once in the \code{FROM} clause.

    The \code{AS} keyword is optional when used in \code{FROM} clauses.
\end{defn}

\begin{defn}{\code{WHERE} clause}
    Pattern match (or with regex) with \code{LIKE} and \code{NOT LIKE} as a condition:
    \begin{itemize}
        \item an underscore (\code{\_}) matches exactly 1 character,
        \item \code{\%} matches 0 or more characters.
    \end{itemize}

    \begin{lstlisting}
SELECT <attributes>
FROM <tables>
WHERE <attribute> [[NOT] LIKE '<pattern>'];
    \end{lstlisting}

    The logical operators \code{AND}, \code{OR}, and \code{NOT} can be used to combine/invert conditions.
\end{defn}

\subsection{Set Operations}
Without \code{ALL}, duplicate rows are eliminated after \code{UNION}, \code{INTERSECT},
and \code{EXCEPT} (set difference).

With \code{ALL}, for every row $x$ in tables $A$ and $B$ appearing $a$ and $b$ times respectively,
$x$ will appear these many times in the result:
\begin{itemize}
    \keyitem*{\code{UNION}}{$a + b$}
    \keyitem*{\code{INTERSECT}}{$\min(a, b)$}
    \keyitem*{\code{EXCEPT}}{$\max(a - b, 0)$}
\end{itemize}

\subsection{Join Operations}
\begin{lstlisting}
SELECT <attributes>
FROM <r1> [NATURAL] [LEFT / RIGHT / FULL] JOIN
       <r2> [ON <condition>],
... -- more joins, query continues
\end{lstlisting}

If the condition and \code{ON} keyword are omitted, the join is implicitly a natural inner or outer join,
and the \code{NATURAL} keyword is optional.

\subsection{Composition}
A \textbf{scalar subquery} is a query that returns a single value (a table with 1 row and 1 column),
or an empty table which is treated as \code{NULL}.

They are dynamically checked at runtime and can be used as a value in a \code{SELECT}, \code{FROM},
or \code{WHERE} clause.

There are other types of subqueries for dynamically checking if values are in a set or not:
\begin{lstlisting}
...
WHERE -- subsequent lines are mutually exclusive
    <expr.> [NOT] IN (<subquery / values>);
    <expr.> <op.> ALL (<subquery / values>);
    <expr.> <op.> ANY (<subquery / values>);
\end{lstlisting}

For \code{ALL} and \code{ANY}, \code{op} can be any comparison operator like \code{=, <, >, <=, >=, <>},
while \code{IN} uses \code{=} and \code{<>} implicitly (not \code{IS DISTINCT FROM}!).

The subquery must return exactly one column for all the above.

There is also \code{WHERE [NOT] EXISTS (<subquery>)}, which is true if the subquery returns at least one row, which can lead to unusual queries like this:

\begin{lstlisting}
...
WHERE EXISTS (
    SELECT 1 -- any value will do
    FROM <table>
    WHERE <condition>
);
\end{lstlisting}

Unlike the others, the subquery of \code{EXISTS} does not need to return a single column.

\section{Persistent Stored Modules}
\subsection{Functions}
Functions in SQL are used to perform operations on data, and must return at least one value.

\begin{lstlisting}
CREATE OR REPLACE FUNCTION <name> (<params>)
    RETURNS <return type> AS $$
        <body>
        [LIMIT 1;] -- for RECORDs
    $$ LANGUAGE sql;
\end{lstlisting}

Input parameters (constants) are prefixed with \code{IN}, output parameters (uninitialized return values) with \code{OUT}, and input/output parameters (initialized variables) with \code{INOUT}.

The return type can be any arbitrary type, but returning an existing tuple requires the table name as the return type.

However, if the function returns a new tuple, the return type must be a \code{RECORD} or \code{SETOF RECORD}, or equivalently, \code{TABLE}.

Function bodies cannot commit or rollback transactions.

Functions are called using regular \code{SELECT} statements:

\begin{lstlisting}
SELECT <name>(<parameters>);
\end{lstlisting}

\begin{defn}{function variables}
    Variables can be declared in function bodies, and they are local to the function.

    \begin{lstlisting}
    -- ... $$
    DECLARE <variable name> <type> [:= <value>];
    BEGIN <body> END;
    -- $$ ...
\end{lstlisting}
\end{defn}

\begin{defn}{function control flow}
    If statements and loops are available as well.

    \begin{lstlisting}
-- ... $$
    IF <condition> THEN <body>
    [ELSEIF <condition> THEN <body>]
    [ELSE <body>]
    END IF;
    LOOP [EXIT WHEN <condition>;]
        <body>;
    END LOOP;
-- $$ ...
\end{lstlisting}
\end{defn}

\begin{defn}{cursors}
    Cursors are used to iterate over the result of a query.
    \begin{lstlisting}
-- $$ ...
    DECLARE <cursor name> CURSOR FOR <query>;
    r RECORD;
    BEGIN
        OPEN <cursor name>;
        LOOP
            FETCH <cursor name> INTO r;
            EXIT WHEN NOT FOUND;
            <body>;
            [RETURN NEXT;]
        END LOOP;
        CLOSE <cursor name>;
    END
-- $$ ...
\end{lstlisting}

    \code{RETURN NEXT;} is used to insert the current value of \code{r} into the result of the function.

    The function return type must be \code{SETOF RECORD} or \code{TABLE} for this to work.

    Variants of \code{FETCH <variant> FROM} with \code{PRIOR}, \code{FIRST}, \code{LAST}, \code{ABSOLUTE <n>} and \code{RELATIVE <n>} are also available.
\end{defn}

\subsection{Procedures}
Unlike functions, procedures do not return any values, but they can commit or rollback transactions as long as they are not called within a transaction.

Procedures have access to the same variable, control flow and cursor constructs as functions.

\begin{lstlisting}
CREATE OR REPLACE PROCEDURE <name> (<params>)
    RETURNS <return type> AS $$
        <body>
    $$ LANGUAGE sql;
\end{lstlisting}

Since functions do not return values, their parameters are implicitly (and need not be) prefixed with \code{IN}. Nonetheless, \code{INOUT} can be used too.

Procedures are called with the \code{CALL} keyword:

\begin{lstlisting}
CALL <name>(<parameters>);
\end{lstlisting}

\subsection{Triggers}
Triggers are used to execute \textbf{trigger functions} when a certain event occurs.

As the name implies, trigger functions are functions with a return type of \code{TRIGGER}.

To get the database to execute a trigger function, it must be registered as a trigger:

\begin{lstlisting}
CREATE TRIGGER <name>
    -- trigger events, can use multiple with OR
    [<AFTER / BEFORE / INSTEAD OF> -- timing
        <INSERT / UPDATE / DELETE>] -- operation
    ON <table>
    [FOR EACH <ROW / STATEMENT>] -- level
    [WHEN <condition>]
    EXECUTE FUNCTION <function name>(<params>);
\end{lstlisting}

Trigger functions can access the inserted or updated tuple using the \code{NEW} variable, and the replaced or deleted tuple using \code{OLD}.

Other special keywords such as \code{TG\_OP} to get the operation type (e.g. \code{INSERT}) and \code{TG\_TABLE\_NAME}.

\begin{defn}{trigger levels}
    \code{FOR EACH ROW} invokes the trigger function for each row affected by the operation.

    \code{FOR EACH STATEMENT} is used to perform operations on the entire table, and the trigger function is invoked once per operation.

    This means that statement-level triggers cannot access the \code{NEW} and \code{OLD} variables.

    Neither can the statement's operation be prevented by returning \code{NULL}, unless an exception is raised with \code{RAISE EXCEPTION}.
\end{defn}

\code{RAISE NOTICE <message>} can be used to print a message to the console, but is not equivalent to \code{RAISE EXCEPTION <message>}.

\begin{defn}{trigger timing - \code{BEFORE}}
    Returning \code{NULL} will prevent the operation (insertion, updating, deletion) from happening.

    Since the value of \code{OLD} is initialized to \code{NULL}, \code{RETURN OLD} will also prevent the operation from happening \textit{unless} some columns of \code{OLD} are modified.
\end{defn}

\begin{defn}{trigger timing - \code{AFTER}}
    The return value of the trigger function has \textbf{no effect} on the operation
    as it is invoked after the operation has already happened.
\end{defn}

\begin{defn}{trigger timing - \code{INSTEAD OF} (row-level only)}
    This may only be defined on views, intended to perform the operation on the underlying table instead of the view.

    Like \code{BEFORE}, returning \code{NULL} will prevent the operation from happening.
\end{defn}

Triggers are checked after every statement immediately by default, but can be deferred until after the \code{COMMIT} of a transaction.

\begin{lstlisting}
CREATE CONSTRAINT TRIGGER <name>
    AFTER <operation> ON <table>
    FOR EACH ROW
    DEFERRABLE INITIALLY <DEFERRED / IMMEDIATE>
    EXECUTE FUNCTION <function name>(<params>);
\end{lstlisting}

Deferred triggers must be \code{AFTER} operations \code{FOR EACH ROW}.

Since triggers are constraints, any deferrable triggers which are set to \code{INITIALLY IMMEDAITE} can be disabled on a per-transaction basis by changing the trigger constraint.

\begin{lstlisting}
BEGIN TRANSACTION;
SET CONSTRAINTS <trigger name> DEFERRED;
-- ...
COMMIT;
-- trigger is deferred until here
\end{lstlisting}

If there are multiple triggers on the same operation of a table, the following order applies, ties broken by alphabetical order of the trigger name:

\begin{enumerate}
    \item \code{BEFORE ... FOR EACH STATEMENT} triggers
    \item \code{BEFORE ... FOR EACH ROW} triggers
    \item \code{AFTER \phantom{ }... FOR EACH ROW} triggers
    \item \code{AFTER \phantom{ }... FOR EACH STATEMENT} triggers
\end{enumerate}

If a row-level \code{BEFORE} trigger function returns \code{NULL}, then all subsequent triggers on the same row are skipped.