\part{Normal Forms}
A \textbf{normal form} is a definition of minimum requirements to reduce data redundancy and improve data integrity.

Redundancies can lead to \textbf{anomalies} where a primary key attribute violates the constraint of being non-\code{NULL}.

These anomalies can be resolved by \textbf{normalization}.


\section{Functional Dependencies}
A \textbf{functional dependency} is a relationship between two sets of attributes $A$ and $B$ that occur when the value of $B$ can always be determined from the value of $A$.

That is, when $A_1 A_2 \dots A_m \rightarrow B_1 B_2 \dots B_n$, any tuple with the same values for $A_1 A_2 \dots A_m$ will have the same value for $B_1 B_2 \dots B_n$.

\begin{defn}{determining functional dependencies}
    Intuitively, the $\rightarrow$ can be read as "decides" or "determines".

    Given attributes $A$ and $B$, if $A$ is a key of the relation (all unique values), then $A \rightarrow B$.

    Another method is to construct a counter-example with 2 tuples which would violate the functional dependency.

    The attribute with the same value in all rows will be on the LHS of FD, and the other attribute on the RHS.
\end{defn}

We can use \textbf{Armstrong's axioms} and some additional rules (denoted with $^\dagger$) to determine if a functional dependency is valid.

\begin{itemize}
    \keyitem*{augmentation}{if \fd{A}{B} then \fd{AC}{BC} for any $C$}
    \keyitem*{transitivity}{if \fd{A}{B} and \fd{B}{C} then \fd{A}{C}}
    \keyitem*{reflexivity}{any set of attributes $\rightarrow$ subset of the attributes}
    \keyitem*{decomposition$^\dagger$}{if \fd{A}{BC} then \fd{A}{B} and \fd{A}{C}}
    \keyitem*{union$^\dagger$}{if \fd{A}{B} and \fd{A}{C} then \fd{A}{BC}}
\end{itemize}

Using these rules is pretty tedious, and we can use \textbf{closures} to simplify the process.

Given a set of functional dependencies, a closure $\{A_1, A_2, \dots, A_n\}^+$ is the set of all attributes that can be determined from $\{A_1, A_2, \dots, A_n\}$.

\begin{defn}{proving functional dependencies \fd{A}{B}}
    To prove that \fd{A}{B} holds, we must show that $B \in \{A\}^+$.

    To prove that \fd{A}{B} does not hold, we must show that $B \notin \{A\}^+$.
\end{defn}

\begin{defn}{finding keys}
    Given a table $R$ and set of functional dependencies, to find the keys:
    \begin{enumerate}
        \item Consider every subset of attributes in $R$.
        \item Derive the closure of each subset.
        \item Find the superkeys which are the attributes whose closures contain every attribute in $R$.
        \item Find the keys which are the minimal superkeys.
    \end{enumerate}

    Some observations to eliminate some of the tedium:
    \begin{itemize}
        \item If a set of attributes is a key, then any superset cannot be a key (only a superkey).
        \item If an attribute is in \textit{none} of the RHSs of the FDs, then it must be part of the key.
    \end{itemize}
\end{defn}

If an attribute is part of a key, then it is known as a \textbf{prime attribute}.
Otherwise, it is a \textbf{non-prime attribute}.


\section{Boyce-Codd Normal Form (BCNF)}
A table $R$ is in BCNF if every non-trivial decomposed FD has a superkey on its LHS.

BCNF requires that every non-prime attribute is fully functionally dependent on a superkey.

A \textbf{decomposed FD} is an FD with only one attribute on the RHS.
Any non-decomposed FD can be decomposed into a set of decomposed FDs.

A \textbf{non-trivial FD} is an FD where no attribute in the RHS appears in the LHS.

\begin{defn}{finding all non-trivial decomposed FDs}
    Find the closures of every possible subset of attributes in the table, and then pair the LHS with each attribute in the RHS of the closure.

    For example, if $\{ A \}^+ = \{ ABC \}$, then the FDs are \fd{A}{B} and \fd{A}{C}.
\end{defn}

\begin{defn}{determine if a table is in BCNF}
    If there exists a non-trivial decomposed FD \fd{A_1 A_2 \dots, A_k}{B_1} on $R$ such that $\{ A_1 A_2 \dots, A_k \}$ is not a superkey, then the table is not in BCNF.

    This is the \textbf{"more but not all"} condition, since the closure $\{ A_1 A_2 \dots, A_k \}^+$ will contain $B_1$ but not all attributes in $R$.

    $^\dagger$ Any table with 2 or fewer attributes is in BCNF.
\end{defn}

\begin{defn}{table decomposition into BCNF}
    First find a subset of the attributes $X$ in $R$ such that its closure $X^+$ contains more attributes than $X$ but not all attributes in $R$.

    Repeatedly perform binary splits on $R$ into smaller tables $R_1$ and $R_2$ until it is in BCNF.

    $R_1$ will contain all attributes in $X^+$. \\
    $R_2$ will contain all attributes in $X$ and the attributes in $R$ but not in $X^+$.

    On the smaller tables, project the original closures of $R$ onto the smaller tables, eliminating the attributes which are no longer in the smaller tables.
\end{defn}

After decomposition, the original table $R$ can be reconstructed via \textbf{lossless joins} as long as there are common attributes between $R_1$ and $R_2$ which are superkeys on \textit{either} table.

The downside of such BCNF decomposition is that dependencies may not be preserved.


\section{Third Normal Form (3NF)}
3NF is less restrictive than BCNF, and requires that every non-trivial decomposed FD satisfies either of the following conditions:

\begin{enumerate}
    \item The LHS is a superkey.
    \item The RHS is a prime attribute, appearing in a key \\ (not allowed by BCNF).
\end{enumerate}

\begin{defn}{determine if a table is in 3NF}
    Find all the keys of $R$, and check if every non-trivial decomposed FD satisfies either of the conditions above.

    $^\dagger$ Any table with 2 or fewer attributes or where every attribute appears in a key is in 3NF.
\end{defn}

Given a set of FDs $S$, a \textbf{minimal basis} of $S$ is a simplified set of FDs that retains all the dependencies in $S$.

Such minimal bases obey the following conditions:
\begin{enumerate}
    \item Every FD in the basis can be derived from $S$ and vice versa.
    \item Every FD in the basis is non-trivial and decomposed.
    \item No FDs in the basis is redundant.
    \item None of the attributes on the LHS of an FD in the basis is redundant.
\end{enumerate}

\begin{defn*}{finding a minimal basis}
    \begin{enumerate}
        \item Decompose the FDs so that the RHSs only contain 1 attribute.
        \item For each FD \fd{AB}{C}, compute the closures of $A$ and $B$. If $C \in A^+$ or $C \in B^+$, then the FD is redundant and can be removed. \\ (If an FD has more than 2 attributes, e.g. \fd{ABC}{D}, then check every possible subset of the LHS for redundancy.)
        \item For every FD \fd{A}{B}, compute the closure of $A$. For every attribute $C$ in the closure, any other FD which has $C$ on its RHS but not $A$ on its LHS is redundant and can be removed.
    \end{enumerate}
\end{defn*}

\begin{defn}{table decomposition into 3NF}
    Given a table $R$ and a set of FDs $S$, first find a minimal basis.

    Merge FDs with the same LHSs using the union rule, i.e. \fd{A}{B} and \fd{A}{C} becomes \fd{A}{BC}.

    Create a table for each remaining FD using the attributes of both sides of the FD, i.e. \fd{ABC}{D} yields $R'(A, B, C, D)$.

    If none of the tables contain a key of $R$, create a table that contains any key of $R$, i.e. for a table $R(A, B, C, D)$ with key $AC$ decomposed into $R_1(A, B)$ and $R_2(C, D)$, create a new table $R_3(A, C)$.

    Remove redundant tables where the attributes of a table are a subset of another table.
\end{defn}