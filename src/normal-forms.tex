\part{Normal Forms}
A \textbf{normal form} is a definition of minimum requirements to reduce data redundancy and improve data integrity.

Redundancies can lead to \textbf{anomalies} where a primary key attribute violates the constraint of being non-\code{NULL}.

These anomalies can be resolved by \textbf{normalization}.


\section{Functional Dependencies}
A \textbf{functional dependency} is a relationship between two sets of attributes $A$ and $B$ that occur when the value of $B$ can always be determined from the value of $A$.

That is, when $A_1 A_2 \dots A_m \rightarrow B_1 B_2 \dots B_n$, any tuple with the same values for $A_1 A_2 \dots A_m$ will have the same value for $B_1 B_2 \dots B_n$.

\begin{defn}{determining functional dependencies}
    Intuitively, the $\rightarrow$ can be read as "decides" or "determines".

    Given attributes $A$ and $B$, if $A$ is a key of the relation (all unique values), then $A \rightarrow B$.
    
    Another method is to construct a counter-example with 2 tuples which would violate the functional dependency.
    
    The attribute with the same value in all rows will be on the LHS of FD, and the other attribute on the RHS. 
\end{defn}

We can use \textbf{Armstrong's axioms} and some additional rules (denoted with $^\dagger$) to determine if a functional dependency is valid.

\begin{itemize}
    \keyitem*{augmentation}{if \fd{A}{B} then \fd{AC}{BC} for any $C$}
    \keyitem*{transitivity}{if \fd{A}{B} and \fd{B}{C} then \fd{A}{C}}
    \keyitem*{reflexivity}{any set of attributes $\rightarrow$ subset of the attributes}
    \keyitem*{decomposition$^\dagger$}{if \fd{A}{BC} then \fd{A}{B} and \fd{A}{C}}
    \keyitem*{union$^\dagger$}{if \fd{A}{B} and \fd{A}{C} then \fd{A}{BC}}
\end{itemize}

Using these rules is pretty tedious, and we can use \textbf{closures} to simplify the process.

Given a set of functional dependencies, a closure $\{A_1, A_2, \dots, A_n\}^+$ is the set of all attributes that can be determined from $\{A_1, A_2, \dots, A_n\}$.

\begin{defn}{proving functional dependencies \fd{A}{B}}
    To prove that \fd{A}{B} holds, we must show that $B \in \{A\}^+$.

    To prove that \fd{A}{B} does not hold, we must show that $B \notin \{A\}^+$.
\end{defn}

\begin{defn}{finding keys}
    Given a table $R$ and set of functional dependencies, to find the keys:
    \begin{enumerate}
        \item Consider every subset of attributes in $R$.
        \item Derive the closure of each subset.
        \item Find the superkeys which are the attributes whose closures contain every attribute in $R$.
        \item Find the keys which are the minimal superkeys.
    \end{enumerate}

    Some observations to eliminate some of the tedium:
    \begin{itemize}
        \item If a set of attributes is a key, then any superset cannot be a key (only a superkey).
        \item If an attribute is in \textit{none} of the RHSs of the FDs, then it must be part of the key.
    \end{itemize}
\end{defn}

If an attribute is part of a key, then it is known as a \textbf{prime attribute}.
Otherwise, it is a \textbf{non-prime attribute}.